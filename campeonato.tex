\documentclass[11pt]{article}
\usepackage[brazil]{babel}
\usepackage[T1]{fontenc}
\usepackage{t1enc}
\usepackage[utf8]{inputenc}
\usepackage{indentfirst}

\title{\textbf{Luta de Robôs -- Campeonato MC346 -- 2014-1S}}

\date{}
\begin{document}

\maketitle

\section{Descrição do Jogo}

O jogo é jogado por dois jogadores, A e B em em um tabuleiro $m \times n$ onde existem espaços vazios, $X$ muros, $K$ recursos, $R$ robôs pertencentes ao jogador $A$ e $R$ robôs pertencentes ao jogador $B$.

Cada robô começa com o valor inicial de energia igual a 1.

O jogo é inicializado aleatoriamente com os parâmetros respeitando os seguintes limites:
\begin{itemize}
\item{m - 10 a 30}
\item{n - 10 a 30}
\item{X - 5 a 25}
\item{R - 2 a 6}
\item{K - 2 a 10}
\end{itemize}

\subsection{Ações}
O jogo alterna entre ações do jogador A e B, começando pelo jogador A. Uma ação é composta pelo identificador do jogador, a posição do robô e a posição de destino. O Robô pode se mover 1 quadrado para cima, para baixo, para esquerda ou para direita. Se o movimento for válido, o robô escolhido então moverá para a posição desejada. O movimento será inválido se a posição que o robô ocuparia após o movimento estiver fora do tabuleiro, for um muro ou for outro robô controlado pelo jogador.

\subsection{Recursos}
Caso haja um recurso na nova posição do robô, este robô adquire este recurso e soma o valor desse recurso à sua energia.

\subsection{Combate}
Caso haja um robô adversário na nova posição do robô, os dois robôs entram em combate. Em um combate, se os robôs tem a mesma quantidade de energia, ambos são destruídos. Se um robô tem mais energia que o outro, este robô prevalece e o outro robô é destruído. O robô ganhador tem a quantidade de energia igual a que possuía menos a quantidade de energia do robô destruído.

\subsection{Fim do jogo}
O jogo termina caso um dos jogadores ficar sem nenhum robô ou o jogo ultrapasse o número máximo de rodadas (90).

No primeiro caso, ganha o jogador que destrui os robôs adversário.

No segundo caso, conta-se a quantidade de energia dos robôs restantes de cada jogador e aquele que obtiver mais energia é o ganhador. Se ambos os jogadores tiverem a mesma quantidade de energia, aquele que tiver mais robôs é o ganhador. Se os jogadores tiverem o mesmo número de robôs, o jogo termina em empate.

\section{Protocolo de Comunicação}

O processo do jogador se comunica com o processo do jogo através da entrada e saída padrão. O formato inicial da entrada segue o seguinte modelo:
\begin{verbatim}
A
15 21
........................................R1
........................XX......XXR1......
........................XX....R1..........
........B1..........R1....................
................R1........................
..........................................
R1..........R1..R1..A1....XX..............
..XX........XX..R1........................
..................R1......................
XX......XX..........................R1....
..XXA1XX..................................
....XX..XX..XX..................XX....XX..
......................B1..................
......R1..........XX........XX........XX..
..............XX......XX..................
\end{verbatim}

A primeira linha informa ao jogador se ele será o jogador A ou o jogador B.

A segunda linha contém dois inteiros que especificam o tamanho do tabuleiro (m e n).

Depois seguem m linhas contendo $n \times 2$ caracteres ASCII. Cada dupla de caracteres corresponde a um quadrado no tabuleiro.

\begin{itemize}
\item{$..$ $\rightarrow$ espaço vazio}
\item{$XX$ $\rightarrow$ muro}
\item{$Ad$ ou $Bd$ $\rightarrow$ Robô de A ou B, d representa um inteiro. No tabuleiro inicial, d é sempre igual a 1 }
\item{$Rd$ $\rightarrow$ Recurso com energia igual a d. No tabuleiro inicial, d pode variar de 1 a 9}
\end{itemize}

Assim que receber o tabuleiro, o jogador A deverá enviar sua ação. O jogador B só envia sua ação depois de receber a ação do jogador A.

Toda comunicação é feita pela entrada e saída padrão. O programa do jogador deve enviar o seu movimento no formato correto na saída padrão, receber o movimento do jogador adversário da entrada padrão.

Uma ação tem o seguinte formato:
\begin{verbatim}
A 7 15 7 16
\end{verbatim}

O primeiro elemento corresponde ao jogador que efetua a ação (A ou B). Os seguintes 4 elementos são, respectivamente i1, j1, i2, j2, onde (i1,j1) é a posição do robô antes da ação e (i2,j2) é a posição do robô depois da ação.

\textbf{Importante:} i representa a linha de cima para baixo do tabuleiro, começando do zero. j representa a coluna da esquerda para direita no tabuleiro, comelando do zero.

\end{document}
